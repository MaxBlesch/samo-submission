%  LaTeX template for abstract submission for SAMO 2022
%
% First name and name of the speaker.
\speaker{Maximilian}{Blesch}%
%  (put no space here)
% Title of the talk, capitalized.
\title{Robust decision-making under risk and ambiguity*}

% For each author, give the first name, family name, affiliation, and email.
% Ideally, the affiliation and email should fit on a single line.
% No need to put the full snail mailing address.
%  One line per author
\author{ Maximilian }{Blesch}{Berlin School of Economics, Germany}{
maximilian.blesch@hu-berlin.de}
\author{Philipp}{Eisenhauer}{University of Bonn, Germany}{peisenha@uni-bonn.de}


% Type your abstract here.
\abstract{%Include your abstract here. You can give references  like here \cite{ref1} or \cite{ref2}.
%Recall that if you need to define your own macros, it is required to include your name in their definitions. \textbf{(1,000 words and 2 pages maximum)}
Decision-makers often confront uncertainties when determining their course of action. For example, individuals save to cover uncertain medical expenses in old age \cite{French.2014}. Firms set prices in an uncertain competitive environment \cite{Ilut.2020}, and policy-makers face uncertainties about future costs and benefits when voting on climate change mitigation efforts \cite{Barnett.2020}. We consider the situation in which a decision-maker posits a collection of economic models to inform his decision-making process. Each model formalizes the relevant objectives and trade-offs. Within a given model, uncertainty is limited to risk, as a model induces a unique probability distribution over possible future outcomes. In addition, however, there is also ambiguity about the true model \cite{Arrow.1951,Knight.1921}.\\

In this context, we focus on the common practice in economics of estimating a subset of the model parameters outside the model and allowing the decision-makers characterized by the model to treat these point estimates as if they correspond to the true parameters. This approach ignores ambiguity about the true model resulting from the parametric uncertainty of the first-step estimation and opens the door to potential misspecification of the decision problem. As-if decision-makers, those who use the point estimates to inform decisions that would be optimal if the estimates were correct \cite{Manski.2021}, face the risk of serious disappointment about their decisions. The performance of as-if decisions often turns out to be very sensitive to misspecification \cite{Smith.2006}, which is particularly consequential in dynamic models where the impact of erroneous decisions accumulates over time \cite{Mannor.2007}. This danger creates the need for robust decision rules that perform well over a whole range of different models instead of an as-if decision rule that performs best for one particular model. However, increasing robustness, often measured by a performance guarantee under a worst-case scenario, decreases performance in all other scenarios. Striking a balance between the two objectives thus poses a significant challenge.\\

We develop a framework to evaluate as-if and robust decision rules in a decision-theoretic setting by merging insights from the literature on data-driven robust optimization \cite{Bertsimas.2018} and robust Markov decision processes \cite{Ben-Tal.2009} with statistical decision theory \cite{Berger.2010}. We set up a stochastic dynamic investment model in which the decision-maker takes ambiguity about the model's transition dynamics directly into account. Using the Kullback-Leibler divergence \cite{Kullback.1951}, we construct ambiguity sets for the transitions that are statistically meaningful, computationally tractable, and anchored in empirical estimates \cite{Ben-Tal.2013}. Our work brings together and extends research in economics and operations to make econometrics useful for decision-making with models \cite{Manski.2021,Bertsimas.2006}.\\

As an applied example, we revisit \cite{Rust.1987}'s seminal bus replacement problem, which serves as a computational illustration in a variety of settings. In the model, the manager Harold Zurcher implements a maintenance plan for a fleet of buses. He faces uncertainty about the future mileage utilization of the buses. To make his plan, he assumes that the mileage utilization follows an exogenous distribution and uses data on past utilization for its estimation. In the standard as-if analysis, the distribution is estimated in a first step and serves as a plug-in for the true unknown distribution. Harold Zurcher makes decisions as if the estimate is correct and ignores any remaining ambiguity about future mileage utilization. We set up a robust version of the bus replacement problem to directly account for estimation uncertainty and explore the properties and relative performance of alternative decision rules.\\


In econometrics, there is burgeoning interest in assessing the sensitivity of findings to model or moment misspecification. Our work is most closely related to \cite{Jorgensen.2021}, who develops a measure to assess the sensitivity of results by fixing a subset of model parameters prior to estimating the remaining parameters. Our approach differs as we directly incorporate model ambiguity in the design of the decision-making process inside the model and assess the performance of a decision rule within a misspecified decision environment. As such, our focus on ambiguity faced by decision-makers inside economic models draws inspiration from the research program summarized in \cite{Hansen.2016}, which tackles similar concerns with a theoretical focus. We complement recent work by \cite{Saghafian.2018}, who works in a setting similar to ours but does not use statistical decision theory to determine the optimal robust decision rule. In ongoing work, \cite{Eisenhauer.2021} use statistical decision theory to structure policy decisions in light of uncertainty about counterfactual policy predictions due to the remaining parametric uncertainty after the estimation of a model. Unlike \cite{Eisenhauer.2021}, who conducts an ex-post evaluation of alternative policy proposals using decision-theoretic criteria, we perform a proper ex-ante analysis of competing decision rules. We evaluate each rule's performance under all possible parameterizations of the model and directly account for parametric uncertainty in their construction. In operations research, there are only a handful of applied examples in which data-driven robust decision-making is used in a dynamic setting including portfolio allocation \cite{Zymler.2013}, elective admission to hospitals \cite{Meng.2015}, scheduling of liver transplantations \cite{Kaufman.2017}, and the cost-effectiveness of colorectal cancer screening policies \cite{Goh.2018}. To the best of our knowledge, none of these applications evaluates the performance of robust decisions against the as-if alternative in a decision-theoretic framework.

%  If you have references, put them here in a format like below.
%  This can be obtained using BiBTeX with the bib style plain.bst, uncommenting first the two next lines and replacing them by the generated .bbl file
\newpage
\bibliographystyle{plain}
\bibliography{literature.bib}
%
%  Note that this bibliography must be placed inside the abstract.
% \begin{thebibliography}{1}
%
%
% \bibliography{literature.bib}
%
% \end{thebibliography}
}  % End of abstract.
